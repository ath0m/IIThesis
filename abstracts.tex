 \polishabstract {W tej pracy sprawdzam skuteczność wykorzystania metody uczenia bez nadzoru w problemie lokalizowania zmian nowotworowych na zdjęciach wykonanych metodą rezonansu magnetycznego. Bazuje na obserwacji, że zmiany patologiczne są rzadkie i są pewnym odstępstwem od normy. Wykorzystałem w tym celu autoenkoder wariacyjny, który pozwala oszacować prawdopodobieństwo wystąpienia danej próbki, co pozwoliło mi na ich klasyfikowanie. Pomysł przetestowałem na danych syntetycznych, wykorzystując do tego zbiór MNIST. Otrzymane wyniki okazały się być zadowalające i zachęcały do przeprowadzenia dalszych eksperymentów już na danych medycznych. Niestety w tym przypadku nie można było uznać tego za sukces, a według mojej analizy model nauczył się jedynie zwracać uwagę na prostą własność, jaką jest jasność próbki skorelowana z ilością kontrastu użytego podczas obrazowania.}
